\chapter{Conclusiones y mejoras futuras}

En este documento se ha presentado el sistema Copper, un entorno de desarrollo
para la creación y manipulación de escenas 3D basadas en Signed Distance Fields
(SDF). A lo largo de los capítulos, se ha detallado la arquitectura,
implementación y funcionamiento de sus principales componentes, incluyendo la
gestión de controles, la implementación de shaders y el sistema de picking.

Entre las conclusiones más relevantes, se destacan las siguientes:

\begin{itemize}
    \item La utilización de SDF permite una representación eficiente y flexible de la
          geometría 3D, facilitando operaciones complejas como la unión, intersección y
          sustracción de objetos.
    \item La integración de herramientas de interfaz gráfica (ImGui) y gizmos proporciona
          una experiencia de usuario intuitiva y directa para la manipulación de la
          escena.
    \item El sistema de picking implementado permite seleccionar objetos de manera
          precisa, incluso en escenas complejas, mejorando la interactividad del entorno.
\end{itemize}

En cuanto a mejoras futuras, se proponen las siguientes líneas de trabajo:

\begin{itemize}
    \item Optimización del rendimiento del sistema de picking, explorando técnicas como
          el uso de estructuras de datos espaciales (por ejemplo, BVH) para acelerar las
          consultas de intersección.
    \item Ampliación de las capacidades de edición en tiempo real, permitiendo a los
          usuarios modificar propiedades de los objetos SDF de manera más dinámica y
          visual.
    \item Implementación de un sistema de materiales y texturas más avanzado, que permita
          una representación visual más rica y realista de los objetos en la escena.
\end{itemize}

Estas mejoras contribuirán a consolidar a Copper como una herramienta potente y
versátil para la creación de escenas 3D, ampliando sus posibilidades y
mejorando la experiencia del usuario.
