\chapter{Introducción}
El desarrollo de los gráficos tridimensionales por computadora ha sido un campo de investigación activo y en constante evolución desde sus inicios.\\
La capacidad de crear representaciones visuales de objetos y escenas en tres dimensiones ha revolucionado diversas industrias, desde el entretenimiento hasta la medicina y la ingeniería.\\
A medida que la tecnología avanza, también lo hacen las técnicas y herramientas utilizadas para generar gráficos 3D, lo que plantea nuevos desafíos y oportunidades para los investigadores y desarrolladores.

En sus inicios, la generación de gráficos 3D estaba limitada por la capacidad de cómputo y se basaba en \textit{pipelines} gráficos fijos compuestos por etapas de transformación, iluminación y rasterización.\\
Posteriormente, con la llegada de los \textit{shaders} programables en GPU (Nvidia GeForce 3 en 2001), fue posible sustituir los \textit{pipelines} fijos por \textit{pipelines} programables, lo que abrió un abanico de posibilidades para la creación de efectos visuales complejos y personalizados, como los algoritmos no basados en polígonos.\\
Esto impulsó la investigación en técnicas de representación más avanzadas, como el \textit{ray tracing} y, en particular, el \textit{ray marching}.

En el contexto del renderizado basado en funciones implícitas, las \textit{Signed Distance Functions} (SDF) no son una invención reciente, sino que tienen sus raíces en trabajos mucho más antiguos.\\
El concepto de combinar funciones implícitas mediante operaciones booleanas se remonta al trabajo de Ricci en 1972~\cite{Ricci:1973:CGC}, y fue ampliado en 1989 por B.~Wyvill y G.~Wyvill con el modelado de \textit{soft objects}~\cite{wyvill1989}.\\
Ese mismo año, Sandin, Hart y Kauffman aplicaron \textit{ray marching} a SDF para renderizar fractales tridimensionales~\cite{hart1989ray}.\\
Posteriormente, en 1995, Hart documentó de nuevo la técnica, aunque la denominó erróneamente \textit{Sphere Tracing}~\cite{hart1996}.

La popularización moderna de las SDF en el ámbito del \textit{renderizado en tiempo real} se debe en gran parte a la comunidad \textit{demoscene}, especialmente a partir de mediados de la década de 2000.\\
Trabajos como el de Crane (2005) y Evans (2006) introdujeron la idea de restringir el campo a una distancia euclidiana real, mejorando el rendimiento y la calidad visual.\\
Sin embargo, el uso de esta técnica no está tan extendido en aplicaciones de renderizado en tiempo real, a pesar de su potencial para crear gráficos visuales y eficientes.
\section{Motivación}

\section{Descripción del problema}

\section{Objetivos}
