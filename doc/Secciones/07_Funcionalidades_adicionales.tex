\chapter{Funcionalidades adicionales implementadas}

\section{Introducción}

Además de las capacidades principales de modelado y renderizado SDF, Copper
incorpora un conjunto de funcionalidades adicionales que mejoran la experiencia
de usuario, la usabilidad y la flexibilidad del sistema. En este capítulo se
describen y documentan con detalle las funcionalidades que han sido añadidas
sobre la base del código existente, su integración en los distintos módulos y
su justificación técnica.

\section{Gestión de la escena: guardado y carga}

La posibilidad de guardar y cargar escenas permite al usuario almacenar el
estado actual del modelado y recuperarlo posteriormente. Esta funcionalidad
está implementada en el módulo \texttt{Coder}, mediante los métodos
\texttt{saveScene} y \texttt{loadScene} (\texttt{src/core/Coder.cpp},
\texttt{src/core/Coder.h}).
\subsection{Guardar escena}

El método \texttt{saveScene(const std::string\& filename)} serializa todos los
objetos presentes en la escena, incluyendo tipo, posición, tamaño, color,
operación y el identificador único. El formato del archivo es texto plano y
sigue una estructura legible, facilitando la interoperabilidad y la depuración.

\subsection{Cargar escena}

El método \texttt{loadScene(const std::string\& filename)} permite restaurar la
escena a partir de un archivo previamente guardado. El sistema parsea cada
línea, reconstruye los objetos y actualiza su identificador, asegurando la
coherencia y la compatibilidad con versiones futuras.

\subsection{Integración en la interfaz}

La gestión de archivos está directamente integrada en la interfaz gráfica
(\texttt{src/ui/Interfaz.cpp}), mediante el uso de \texttt{ImGuiFileDialog}. El
usuario puede seleccionar el archivo deseado para guardar o cargar la escena, y
el sistema actualiza la visualización en tiempo real. Este flujo está soportado
por el siguiente fragmento:

\begin{verbatim}
if (ImGuiFileDialog::Instance()->IsOk()) {
    std::string filePath = ImGuiFileDialog::Instance()->GetFilePathName();
    coder->saveScene(filePath);
}
\end{verbatim}

\section{Gestión de la luz}

Copper permite modificar la posición de la fuente de luz principal en la
escena. Esta funcionalidad se encuentra en la interfaz gráfica y en el módulo
\texttt{Renderer} (\texttt{src/core/Renderer.cpp},
\texttt{src/core/Renderer.h}).

\subsection{Control de la luz desde la GUI}

La posición de la luz puede ser ajustada mediante un slider en ImGui:

\begin{verbatim}
static float lightPos[3] = {0.0f, 5.0f, 0.0f};
if (ImGui::SliderFloat3("Light Position", lightPos, -20.0f, 20.0f)) {
    renderer->setLightPosition(lightPos[0], lightPos[1], lightPos[2]);
}
\end{verbatim}

\subsection{Actualización del pipeline}

El método \texttt{setLightPosition} actualiza el valor en los uniforms del
renderer, permitiendo que el cálculo de iluminación se adapte a la nueva
posición en tiempo real. Esto afecta directamente al sombreado en los shaders.

\section{Visualización y edición de objetos SDF}

La interfaz gráfica permite visualizar, editar y eliminar objetos SDF desde el
panel de objetos. El usuario puede modificar posición, color, tamaño y tipo de
operación de cada objeto.

\subsection{Edición en tiempo real}

Al seleccionar un objeto, se muestran controles específicos según el tipo
(esfera, caja, cono, cilindro). Los cambios realizados se reflejan
inmediatamente en la escena, marcando el pipeline como "dirty" para que el
renderer actualice el renderizado.

\begin{verbatim}
ImGui::SliderFloat("Radius", &selectedObject.size[0], 0.1f, 5.0f);
ImGui::ColorEdit3("Color", &selectedObject.r);
\end{verbatim}

\subsection{Eliminación y gestión de objetos}

La interfaz incluye un botón "Delete Object" que elimina el objeto
seleccionado. El método \texttt{deleteObject} ajusta el identificador y
actualiza la escena.

\section{Control de operaciones entre objetos}

Copper soporta operaciones booleanas y suaves entre objetos: unión,
intersección, sustracción, smooth union, smooth subtract. El usuario selecciona
la operación desde la GUI y el sistema actualiza el shader generado.

\subsection{Integración y propagación de cambios}

El cambio de operación en la interfaz se propaga al objeto y marca el pipeline
como "dirty". El método \texttt{generateShaderCode} en \texttt{Coder.cpp} añade
la operación correspondiente en el código WGSL, permitiendo combinaciones
arbitrarias.

\section{Mostrar/ocultar el plano de referencia (floor)}

El plano de referencia (floor) puede activarse o desactivarse desde la interfaz
gráfica. El parámetro \texttt{renderer->floor} controla si se incluye el plano
en la generación del shader y en el renderizado. Esta funcionalidad ayuda a
mejorar la visualización y ajuste de los objetos en la escena.

\section{Visualización de FPS y métricas}

La interfaz muestra el número de frames por segundo (FPS) utilizando
\texttt{ImGui::GetIO().Framerate}, permitiendo valorar el rendimiento del
sistema durante la edición y el renderizado.
