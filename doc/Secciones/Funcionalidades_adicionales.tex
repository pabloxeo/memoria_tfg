\chapter{Funcionalidades adicionales implementadas}

\section{Introducción}

Además de las capacidades principales de modelado y renderizado SDF, Copper
incorpora un conjunto de funcionalidades adicionales que mejoran la experiencia
de usuario, la usabilidad y la flexibilidad del sistema. En este capítulo se
describen y documentan con detalle las funcionalidades que han sido añadidas
sobre la base del código existente y su integración en los distintos módulos.

\section{Gestión de la escena: guardado y carga}

La posibilidad de guardar y cargar escenas permite al usuario almacenar el
estado actual del modelado y recuperarlo posteriormente. Esta funcionalidad
está implementada en el módulo \texttt{Coder}, mediante los métodos
\texttt{saveScene} y \texttt{loadScene} (\texttt{src/core/Coder.cpp},
\texttt{src/core/Coder.h}).
\subsection{Guardar escena}

El método \texttt{saveScene(const std::string\& filename)} serializa todos los
objetos presentes en la escena, incluyendo tipo, posición, tamaño, color,
operación y el identificador único. El formato del archivo es texto plano y
sigue una estructura legible, facilitando la interoperabilidad y la depuración.

\subsection{Cargar escena}

El método \texttt{loadScene(const std::string\& filename)} permite restaurar la
escena a partir de un archivo previamente guardado. El sistema parsea cada
línea, reconstruye los objetos y actualiza su identificador, asegurando la
coherencia y la compatibilidad con versiones futuras.

\subsection{Integración en la interfaz}

La gestión de archivos está directamente integrada en la interfaz gráfica
(\texttt{src/ui/Interfaz.cpp}), mediante el uso de \texttt{ImGuiFileDialog}. El
usuario puede seleccionar el archivo deseado para guardar o cargar la escena, y
el sistema actualiza la visualización en tiempo real. Este flujo está soportado
por el siguiente fragmento:

\begin{verbatim}
if (ImGuiFileDialog::Instance()->IsOk()) {
    std::string filePath = ImGuiFileDialog::Instance()->GetFilePathName();
    coder->saveScene(filePath);
}
\end{verbatim}


\section{Visualización de FPS y métricas}

La interfaz muestra el número de frames por segundo (FPS) utilizando
\texttt{ImGui::GetIO().Framerate}, permitiendo valorar el rendimiento del
sistema durante la edición y el renderizado.

\subsection{Pruebas de rendimiento con \texttt{addTest}}

Se han realizado pruebas de rendimiento utilizando el método \texttt{addTest} de
\texttt{Coder}, el cual añade un conjunto de esferas a la escena para evaluar
el comportamiento del sistema bajo diferentes cargas de objetos. Los resultados
obtenidos, variando el número de objetos y la presencia del plano de suelo, son
los siguientes:

\begin{itemize}
    \item 20 objetos: \\
        \textbf{Sin suelo:} 240 FPS \\
        \textbf{Con suelo:} 120 FPS
    \item 50 objetos: \\
        \textbf{Sin suelo:} 180 FPS \\
        \textbf{Con suelo:} 60 FPS
    \item 100 objetos: \\
        \textbf{Sin suelo:} 100 FPS \\
        \textbf{Con suelo:} 30 FPS
    \item 200 objetos: \\
        \textbf{Sin suelo:} 25 FPS \\
        \textbf{Con suelo:} 10 FPS
\end{itemize}

Estos resultados muestran cómo el rendimiento (FPS) se ve afectado por el número
de objetos y la presencia del suelo en la escena. Se observa una disminución
progresiva de los FPS a medida que aumenta el número de objetos, siendo más
acusada cuando se activa el plano de suelo, esto es causado principalmente por la forma actual de calcular las sombras.

Las especificaciones del sistema utilizado para realizar estas pruebas son las siguientes:

\begin{itemize}
    \item CPU: AMD Ryzen 5 5600X
    \item GPU: AMD Radeon RX 6800
    \item RAM: 32 GB
    \item Sistema operativo: Arch Linux
\end{itemize}
