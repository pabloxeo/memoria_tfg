\chapter{Introducción}
El desarrollo de los gráficos tridimensionales por computadora ha sido un campo
de investigación activo y en constante evolución desde sus inicios.\\ La
capacidad de crear representaciones visuales de objetos y escenas en tres
dimensiones ha revolucionado diversas industrias, desde el entretenimiento
hasta la medicina y la ingeniería.\\ A medida que la tecnología avanza, también
lo hacen las técnicas y herramientas utilizadas para generar gráficos 3D, lo
que plantea nuevos desafíos y oportunidades para los investigadores y
desarrolladores.

En sus inicios, la generación de gráficos 3D estaba limitada por la capacidad
de cómputo y se basaba en \textit{pipelines} gráficos fijos compuestos por
etapas de transformación, iluminación y rasterización.\\ Posteriormente, con la
llegada de los \textit{shaders} programables en GPU (Nvidia GeForce 3 en 2001),
fue posible sustituir los \textit{pipelines} fijos por \textit{pipelines}
programables, lo que abrió un abanico de posibilidades para la creación de
efectos visuales complejos y personalizados, como los algoritmos no basados en
polígonos.\\ Esto impulsó la investigación en técnicas de representación más
avanzadas, como el \textit{ray tracing} y, en particular, el \textit{ray
    marching}.

En el contexto del renderizado basado en funciones implícitas, las
\textit{Signed Distance Functions} (SDF) no son una invención reciente, sino
que tienen sus raíces en trabajos mucho más antiguos.\\ El concepto de combinar
funciones implícitas mediante operaciones booleanas se remonta al trabajo de
Ricci en 1972~\cite{Ricci:1973:CGC}, y fue ampliado en 1989 por B.~Wyvill y
G.~Wyvill con el modelado de \textit{soft objects}~\cite{wyvill1989}.\\ Ese
mismo año, Sandin, Hart y Kauffman aplicaron \textit{ray marching} a SDF para
renderizar fractales tridimensionales~\cite{hart1989ray}.\\ Posteriormente, en
1995, Hart documentó de nuevo la técnica, a la que denominó \textit{Sphere
    Tracing}~\cite{hart1996}.

La popularización moderna de las SDF en el ámbito del \textit{renderizado en
    tiempo real} se debe en gran parte a la comunidad \textit{demoscene},
especialmente a partir de mediados de la década de 2000.\\ Trabajos como el de
Crane (2005) y Evans (2006) introdujeron la idea de restringir el campo a una
distancia euclidiana real, mejorando el rendimiento y la calidad visual.\\ Sin
embargo, el uso de esta técnica no está tan extendido en aplicaciones de
renderizado en tiempo real, a pesar de su potencial para crear gráficos
visuales y eficientes.
\section{Motivación}

Como cualquier persona nacida en los dos mil, he crecido rodeado de videojuegos
y el avance en la tecnología de gráficos 3D ha sido un aspecto fascinante de
esta industria. Durante la carrera de informática, estudié diversas asignaturas
relacionadas con gráficos por computadora, cuyos proyectos despertaron y
consolidaron mi interés en este campo.

El desarrollo de motores gráficos y técnicas de renderizado siempre me ha
resultado un área especialmente atractiva, no solo por su complejidad técnica,
sino también por el impacto directo que tienen en sectores como el
entretenimiento, la simulación o la realidad virtual. A lo largo de mis
estudios me encontré con herramientas muy potentes, pero también con la
dificultad que implica dominarlas o adaptarlas a entornos experimentales. Esto
me llevó a plantearme la posibilidad de crear una aplicación propia que
sirviera como espacio de exploración.

La motivación principal de este trabajo es profundizar en tecnologías
emergentes, en concreto \textbf{WebGPU}, un estándar reciente que promete
unificar el desarrollo gráfico multiplataforma con un acceso eficiente a las
GPU modernas. Asimismo, me interesaba experimentar con el modelado mediante
\textbf{funciones de distancia (SDF)}, que representan una alternativa flexible
al modelado poligonal clásico. Considero que la combinación de ambas
tecnologías constituye un terreno de investigación con un gran potencial, tanto
en aplicaciones prácticas como en entornos educativos.

Finalmente, este proyecto me ofrece la oportunidad de afianzar mis
conocimientos en programación gráfica, shaders y arquitecturas modernas de GPU,
a la vez que desarrollo un software propio que pueda servir de base para
futuros trabajos de investigación o aplicaciones más complejas en el ámbito del
diseño 3D.

\section{Descripción del problema}

El campo del modelado y renderizado 3D ha estado tradicionalmente dominado por
herramientas complejas y de gran envergadura, como Blender, Maya o 3ds Max. Si
bien estas aplicaciones ofrecen una gran potencia y versatilidad, presentan
también limitaciones importantes: requieren elevados recursos de hardware,
poseen curvas de aprendizaje pronunciadas y no siempre resultan adecuadas para
entornos de experimentación ligera o proyectos educativos.

Por otro lado, las API gráficas más extendidas, como OpenGL o DirectX, han
demostrado su eficacia a lo largo de los años, pero presentan restricciones en
cuanto a eficiencia y portabilidad en plataformas modernas. El reciente
estándar \textbf{WebGPU} surge como respuesta a estas carencias, ofreciendo un
modelo de programación más cercano al hardware y multiplataforma, con el
objetivo de unificar el desarrollo gráfico en navegadores y aplicaciones
nativas.

En el ámbito del modelado, el paradigma poligonal sigue siendo el más
utilizado, pero alternativas como las \textbf{funciones de distancia (SDF)}
permiten representar geometrías complejas de manera más compacta y flexible,
facilitando la combinación de primitivas y operaciones booleanas. Sin embargo,
la integración de estas técnicas en aplicaciones prácticas todavía es limitada,
especialmente en combinación con tecnologías emergentes como WebGPU.

El problema que aborda este trabajo consiste en la falta de herramientas
ligeras que sirvan como demostración y entorno de experimentación para el
modelado y renderizado basados en SDF sobre WebGPU.

\section{Objetivos}

\subsection*{Objetivo general}
El objetivo principal de este Trabajo de Fin de Grado es el desarrollo de una
aplicación de diseño 3D basada en \textbf{WebGPU} y en técnicas de
\textbf{modelado mediante funciones de distancia (SDF)}, que permita explorar y
demostrar el potencial de estas tecnologías como alternativa al modelado
poligonal clásico y como herramienta de experimentación en el ámbito de los
gráficos por computadora.

\subsection*{Objetivos específicos}
Para alcanzar este objetivo general, se plantean los siguientes objetivos
específicos:

\begin{itemize}
    \item Investigar y comprender en profundidad el funcionamiento de la API WebGPU y su
          integración en aplicaciones nativas mediante la librería Dawn.
    \item Diseñar e implementar un motor de renderizado basado en \textit{ray marching}
          sobre funciones de distancia, capaz de representar primitivas y combinaciones
          mediante operaciones booleanas y suaves.
    \item Incorporar técnicas de sombreado y efectos visuales (iluminación, sombras
          suaves) que mejoren la calidad del renderizado.
    \item Desarrollar una interfaz gráfica sencilla que permita al usuario interactuar
          con la escena y manipular las primitivas.
\end{itemize}

\section{Estructura de la memoria}

La presente memoria se organiza en los siguientes capítulos:

\begin{itemize}
    \item \textbf{Introducción:} Se expone el contexto del trabajo, la motivación, los objetivos planteados y la justificación de la elección de las tecnologías empleadas.
    \item \textbf{Fundamentos teóricos:} Se revisan los conceptos clave de modelado y renderizado 3D, las funciones de distancia (\textit{Signed Distance Functions, SDF}) y el estándar WebGPU, contextualizando el trabajo en el estado actual de la tecnología.
    \item \textbf{Arquitectura de la aplicación:} Se describe la estructura general de la aplicación Copper, detallando los principales módulos, componentes y la interacción entre ellos.
    \item \textbf{Implementación del módulo Ventana:} Se detalla el desarrollo del módulo encargado de la gestión de la ventana y los eventos de usuario, incluyendo la integración con GLFW y la configuración inicial.
    \item \textbf{Implementación y diseño de los controles:} Se aborda el desarrollo de la interfaz de usuario, incluyendo la disposición de los elementos, la gestión de eventos y la interacción con la escena 3D.
    \item \textbf{Implementación de los shaders:} Se describe el desarrollo del generador de código de WGSL, y su actualización dinámica para controlar la escena.
    \item \textbf{Implementación de la cámara:} Se detalla la construcción de la matriz de vista y la gestión de la posición y orientación de la cámara en el espacio 3D.
    \item \textbf{Funcionalidades adicionales implementadas:} Se documentan las características extra añadidas a la aplicación, como el sistema de guardado/carga de escenas y la gestión de la luz.
    \item \textbf{Conclusiones y trabajos futuros:} Se realiza un balance del trabajo realizado, se revisa el grado de cumplimiento de los objetivos y se plantean posibles líneas de investigación y desarrollo futuras.
    \item \textbf{Bibliografía y anexos:} Se recopilan las referencias bibliográficas consultadas y se incluyen materiales complementarios relevantes para la comprensión y reproducibilidad del trabajo.
\end{itemize}