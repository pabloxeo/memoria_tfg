\thispagestyle{empty}

\begin{center}
{\large\bfseries Desarrollo software de una aplicación de modelado 3d basada en Signed Distance Funtion (sdf) \\ Motor de renderizado en tiempo real usando Dawn, C++ y WebGPU }\\
\end{center}
\begin{center}
Pablo Cantudo Gómez\\
\end{center}

\vspace{0.7cm}

\vspace{0.5cm}
\noindent\textbf{Palabras clave}: \textit{SDF}, \textit{WebGPU}, \textit{C++}, \textit{ray marching}, \textit{motor de renderizado}
\vspace{0.7cm}

\noindent\textbf{Resumen}\\
\break
Este trabajo presenta el desarrollo de \textit{Copper}, un motor de renderizado 3D en tiempo real implementado en C++ utilizando la API WebGPU a trav\'es de Dawn.
El motor emplea t\'ecnicas de \textit{ray marching} sobre funciones de distancia con signo (SDF), lo que permite una representaci\'on eficiente y flexible de geometr\'ia 
 tridimensional. Una de las principales ventajas de utilizar SDF es la capacidad de expresar de forma directa y compacta operaciones booleanas entre primitivas geom\'etricas
  ---como uniones, intersecciones o diferencias--- mediante simples expresiones algebraicas. Adem\'as, a diferencia de los m\'etodos tradicionales basados en mallas o 
  pol\'igonos, las SDF permiten aplicar operaciones booleanas suaves, como la uni\'on con suavizado (\textit{smooth union}), 
  de forma pr\'acticamente trivial desde el punto de vista computacional.
\bigbreak
Este enfoque simplifica notablemente la construcci\'on de formas complejas, evitando problemas t\'ipicos de la computaci\'on geom\'etrica como el manejo 
de v\'ertices, normales o topolog\'ias complejas. Tambi\'en se facilita la animaci\'on y transformaci\'on de objetos mediante funciones continuas. El motor 
incluye un sistema de sombreado en WGSL, con soporte para sombras suaves, operaciones modulares sobre la escena, seleccion de objetos, carga y guardado de modelos. Los resultados demuestran que
el uso de SDF no solo ofrece un modelo m\'as elegante para definir geometr\'ia, sino que permite construir escenas visualmente complejas con menos c\'odigo y una mayor expresividad gr\'afica.
En conjunto, el sistema demuestra la viabilidad t\'ecnica y creativa del uso de SDF en entornos modernos.
\cleardoublepage

\begin{center}
	{\large\bfseries Development of a 3D modeling application based on Signed Distance Functions (SDF) \\ Real-time rendering engine using Dawn, C++ and WebGPU}\\
\end{center}
\begin{center}
	Pablo Cantudo Gómez\\
\end{center}
\vspace{0.5cm}
\noindent\textbf{Keywords}: \textit{SDF}, \textit{WebGPU}, \textit{C++}, \textit{ray marching}, \textit{rendering engine}
\vspace{0.7cm}

\noindent\textbf{Abstract}\\
\break
This work presents the development of Copper, a real-time 3D rendering engine implemented in C++ using the WebGPU API through Dawn.
The engine employs ray marching techniques over Signed Distance Functions/Fields (SDF), enabling an efficient and flexible representation of three-dimensional geometry.
One of the main advantages of using SDF is the ability to directly and compactly express Boolean operations between geometric primitives ---such as unions, intersections, or differences--- through simple algebraic expressions.
Furthermore, unlike traditional mesh- or polygon-based methods, SDF allows for smooth Boolean operations, such as smooth union, in a virtually trivial manner from a computational standpoint.
\bigbreak
This approach greatly simplifies the construction of complex shapes, avoiding typical problems in geometric computing such as handling vertices, normals, or complex topologies.
It also facilitates the animation and transformation of objects through continuous functions.
The engine includes a shading system in WGSL, with support for soft shadows, modular scene operations, object selection, and model loading and saving.
The results show that the use of SDF not only offers a more elegant model for defining geometry, but also enables the creation of visually complex scenes with less code and greater graphical expressiveness.
Overall, the system demonstrates the technical and creative feasibility of using SDF in modern environments.

\cleardoublepage

\thispagestyle{empty}

\noindent\rule[-1ex]{\textwidth}{2pt}\\[4.5ex]

D. \textbf{Tutora/e(s)}, Profesor(a) del ...

\vspace{0.5cm}

\textbf{Informo:}

\vspace{0.5cm}

Que el presente trabajo, titulado \textit{\textbf{Chief}},
ha sido realizado bajo mi supervisión por \textbf{Estudiante}, y autorizo la defensa de dicho trabajo ante el tribunal
que corresponda.

\vspace{0.5cm}

Y para que conste, expiden y firman el presente informe en Granada a Junio de 2018.

\vspace{1cm}

\textbf{El/la director(a)/es: }

\vspace{5cm}

\noindent \textbf{(nombre completo tutor/a/es)}

\chapter*{Agradecimientos}
Quiero expresar mi agradecimiento a todas las personas que han contribuido directa o indirectamente a la realización de este trabajo.\break

En primer lugar, a mi tutor Juan Carlos Torres Cantero y cotutor Luis López Escudero por la ayuda durante el desarrollo y la correci\'on del proyecto,
a mis amigos por las discusiones y las ideas, y a mi familia por su apoyo incondicional.\break

 Finalmente, me gustaría mencionar a la comunidad de desarrolladores y recursos abiertos, especialmente los foros, artículos y proyectos relacionados con WebGPU, SDF y ray marching,
 cuya documentación y ejemplos han sido una fuente de aprendizaje invaluable.



