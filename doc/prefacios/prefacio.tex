\chapter*{}
%\thispagestyle{empty}
%\cleardoublepage

%\thispagestyle{empty}

\cleardoublepage
\thispagestyle{empty}

\begin{center}
{\large\bfseries Desarrollo software de una aplicación de modelado 3D
basada en Signed Distance Functions (SDF)}\\
\end{center}
\begin{center}
Pablo Cantudo Gómez\\
\end{center}

%\vspace{0.7cm}
\noindent{\textbf{Palabras clave}: WebGPU, C++, ray marching, motor de renderizado, funciones de distancia con signo (SDF)}\\

\vspace{0.7cm}
\noindent{\textbf{Resumen}}\\

Este trabajo presenta el desarrollo de Copper, un motor de renderizado 3D en
tiempo real implementado en C++ utilizando la API WebGPU a través de
Dawn. El motor emplea técnicas de ray marching sobre funciones de distancia
con signo (SDF), lo que permite una representación eficiente y flexible de
geometría tridimensional. Una de las principales ventajas de utilizar SDF es
la capacidad de expresar de forma directa y compacta operaciones booleanas
entre primitivas geométricas —como uniones, intersecciones o diferencias—
mediante simples expresiones algebraicas. Además, a diferencia de los métodos
tradicionales basados en mallas o polígonos, las SDF permiten aplicar
operaciones booleanas suaves, como la unión con suavizado (smooth union),
de forma prácticamente trivial desde el punto de vista computacional.
\bigbreak
Este enfoque simplifica notablemente la construcción de formas complejas,
 evitando problemas típicos de la computación geométrica como el
manejo de vértices, normales o topologías complejas. También se facilita
la animación y transformación de objetos mediante funciones continuas. El
motor incluye un sistema de sombreado en WGSL, con soporte para sombras
suaves, operaciones modulares sobre la escena, seleccion de objetos, carga
y guardado de modelos. Los resultados demuestran que el uso de SDF no
solo ofrece un modelo más elegante para definir geometría, sino que permite
construir escenas visualmente complejas con menos código y una mayor expresividad gráfica.
En conjunto, el sistema demuestra la viabilidad técnica
y creativa del uso de SDF en entornos modernos.
\cleardoublepage


\thispagestyle{empty}


\begin{center}
{\large\bfseries Development of a 3D modeling application based on Signed
Distance Functions/Fields (SDF)}\\
\end{center}
\begin{center}
Pablo Cantudo Gomez \\
\end{center}

%\vspace{0.7cm}
\noindent{\textbf{Keywords}: WebGPU, C++, ray marching, rendering engine, Signed Distance Functions/Fields (SDF)}\\

\vspace{0.7cm}
\noindent{\textbf{Abstract}}\\

This work presents the development of Copper, a realtime 3D rendering engine implemented in C++ using the WebGPU API through Dawn. The engine employs ray marching techniques over Signed Distance Functions/Fields (SDF), enabling an efficient and flexible representation of three-dimensional geometry. One of the main advantages of using SDF is the ability to directly and compactly express Boolean operations between geometric primitives —
such as unions, intersections, or differences— through simple algebraic expressions. Furthermore, unlike traditional mesh or polygon based methods,
SDF allows for smooth Boolean operations, such as smooth union, in a virtually trivial manner from a computational standpoint.
\bigbreak
This approach greatly simplifies the construction of complex shapes,
avoiding typical problems in geometric computing such as handling vertices, normals, or complex topologies. It also facilitates the animation and
transformation of objects through continuous functions. The engine includes
a shading system in WGSL, with support for soft shadows, modular scene
operations, object selection, and model loading and saving. The results show
that the use of SDF not only offers a more elegant model for defining geometry, but also enables the creation of visually complex scenes with less code
and greater graphical expressiveness. Overall, the system demonstrates the
technical and creative feasibility of using SDF in modern environments.

\chapter*{}
\thispagestyle{empty}

\noindent\rule[-1ex]{\textwidth}{2pt}\\[4.5ex]

Yo, \textbf{Pablo Cantudo Gómez}, alumno de la titulación Grado en ingeniería informática de la \textbf{Escuela Técnica Superior
de Ingenierías Informática y de Telecomunicación de la Universidad de Granada}, con DNI 78243264, autorizo la
ubicación de la siguiente copia de mi Trabajo Fin de Grado en la biblioteca del centro para que pueda ser
consultada por las personas que lo deseen.

\vspace{6cm}

\noindent Fdo: Pablo Cantudo Gómez

\vspace{2cm}

\begin{flushright}
Granada a x de Septiembre de 2025.
\end{flushright}


\chapter*{}
\thispagestyle{empty}

\noindent\rule[-1ex]{\textwidth}{2pt}\\[4.5ex]

D. \textbf{Tutores}, Profesores del Área de x del Departamento x de la Universidad de Granada.

\vspace{0.5cm}

\textbf{Informan:}

\vspace{0.5cm}

Que el presente trabajo, titulado \textit{\textbf{Nombre}},
ha sido realizado bajo su supervisión por \textbf{Tutores}, y autorizamos la defensa de dicho trabajo ante el tribunal
que corresponda.

\vspace{0.5cm}

Y para que conste, expiden y firman el presente informe en Granada a 11 de Noviembre de 2023.

\vspace{1cm}

\textbf{Los directores:}

\vspace{5cm}

\noindent \textbf{Tutores}

\chapter*{Agradecimientos}
\thispagestyle{empty}

       \vspace{1cm}


Quiero expresar mi agradecimiento a todas las personas que han contribuido directa o indirectamente a la realización de este trabajo.
En primer lugar, a mi tutor Juan Carlos Torres Cantero y cotutor Luis
López Escudero por la ayuda durante el desarrollo y la corrección del proyecto, a mis amigos por las discusiones y las ideas, y a mi familia por su apoyo incondicional.
\bigbreak
Finalmente, me gustaría mencionar a la comunidad de desarrolladores y
recursos abiertos, especialmente los foros, artículos y proyectos relacionados
con WebGPU, SDF y ray marching, cuya documentación y ejemplos han
sido una fuente de aprendizaje de gran valor.