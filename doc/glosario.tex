% --- Definiciones para el glosario ---
\newglossaryentry{sdf}{
    name={Signed Distance Function (SDF)},
    description={Función que, dado un punto en el espacio, devuelve la distancia mínima a la superficie más cercana, con signo positivo si el punto está fuera y negativo si está dentro.}
}

\newglossaryentry{raymarching}{
    name={Ray Marching},
    description={Técnica de renderizado que recorre un rayo en pasos discretos para encontrar intersecciones con superficies definidas implícitamente.}
}

\newglossaryentry{csg}{
    name={Constructive Solid Geometry (CSG)},
    description={Método de modelado geométrico que combina primitivas mediante operaciones booleanas como unión, intersección y diferencia.}
}

\newglossaryentry{smoothblending}{
    name={Smooth Blending},
    description={Técnica para suavizar las transiciones entre primitivas geométricas en modelado implícito, evitando uniones abruptas.}
}

\newglossaryentry{demoscene}{
    name={Demoscene},
    description={Comunidad de programadores, artistas y músicos que crean producciones audiovisuales en tiempo real para mostrar destreza técnica y creatividad.}
}

\newglossaryentry{implicitfunction}{
    name={Función Implícita},
    description={Función matemática que describe una superficie como el conjunto de puntos que satisfacen una ecuación dada, sin necesidad de una parametrización explícita.}
}

\newglossaryentry{sphereTracing}{
    name={Sphere Tracing},
    description={Método propuesto por Hart en 1995 para recorrer campos de distancia en el renderizado de superficies implícitas.}
}
\newglossaryentry{hypertexture}{
    name={Hypertexture},
    description={Técnica de Ken Perlin y Louis Hoffert para renderizar volúmenes procedurales, que sirvió de base para el desarrollo del ray marching.}
}

\newglossaryentry{gizmo}{
    name={Gizmo},
    description={Elemento gráfico interactivo en una interfaz que permite manipular objetos o parámetros visualmente, como mover, rotar o escalar en aplicaciones de diseño o gráficos 3D.}
}

\newglossaryentry{webgpu}{
    name={WebGPU},
    description={Estándar gráfico moderno que proporciona acceso eficiente y multiplataforma a la GPU, tanto en navegadores como en aplicaciones nativas.}
}

\newglossaryentry{shader}{
    name={Shader},
    description={Programa que se ejecuta en la GPU para transformar vértices, calcular colores y simular efectos visuales en el renderizado.}
}

\newglossaryentry{wgsl}{
    name={WGSL},
    description={WebGPU Shading Language, lenguaje nativo de WebGPU para escribir shaders y operaciones matemáticas avanzadas.}
}

\newglossaryentry{pipeline}{
    name={Pipeline},
    description={Secuencia de etapas de procesamiento en la GPU para renderizar gráficos, desde la entrada de vértices hasta la salida de píxeles.}
}

\newglossaryentry{picking}{
    name={Picking},
    description={Técnica para identificar y seleccionar objetos en una escena gráfica mediante la posición del cursor o eventos de usuario.}
}

\newglossaryentry{imgui}{
    name={ImGui},
    description={Biblioteca de interfaz gráfica inmediata utilizada para crear menús, controles y herramientas interactivas en aplicaciones gráficas.}
}

\newglossaryentry{glm}{
    name={GLM},
    description={OpenGL Mathematics, biblioteca para operaciones matemáticas con vectores, matrices y cuaterniones en gráficos 3D.}
}

\newglossaryentry{glfw}{
    name={GLFW},
    description={Biblioteca multiplataforma para la gestión de ventanas, entrada de usuario y eventos en aplicaciones gráficas.}
}

\newglossaryentry{cuaternion}{
    name={Cuaternión},
    description={Estructura matemática utilizada para representar rotaciones en el espacio tridimensional, evitando problemas como el gimbal lock.}
}

\newglossaryentry{framerate}{
    name={Framerate},
    description={Número de imágenes (frames) renderizadas por segundo, indicador del rendimiento de una aplicación gráfica.}
}

\newglossaryentry{renderer}{
    name={Renderer},
    description={Módulo o componente encargado de gestionar el proceso de renderizado de la escena y la comunicación con la GPU.}
}

\newglossaryentry{callback}{
    name={Callback},
    description={Función que se ejecuta en respuesta a un evento, como la pulsación de un botón o el redimensionado de una ventana.}
}

\newglossaryentry{uniform}{
    name={Uniform},
    description={Variable global enviada desde la CPU al shader, utilizada para transmitir parámetros como matrices, colores o posiciones.}
}
